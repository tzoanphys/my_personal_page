\documentclass[a4paper,11pt]{article}

%--- Packages ---
\usepackage[utf8]{inputenc}
\usepackage[T1]{fontenc}
\usepackage{geometry}
\geometry{top=2.5cm, bottom=2.5cm, left=2.5cm, right=2.5cm}
\usepackage{amsmath, amssymb, amsfonts}
\usepackage{graphicx}
\usepackage{hyperref}
\usepackage{parskip}
\usepackage{booktabs} % For nicer tables if needed

%--- Color Definitions ---
\usepackage{xcolor}
\definecolor{primaryTeal}{RGB}{0, 128, 128}    % Main Header Color
\definecolor{secondaryGray}{RGB}{80, 80, 80}   % Subheaders / Text
\definecolor{boxBack}{RGB}{245, 250, 250}      % Light background for boxes
\definecolor{accentColor}{RGB}{0, 100, 100}    % Darker teal for emphasis

%--- Font and Header Styling ---
\usepackage{sectsty}
\sectionfont{\color{primaryTeal}\LARGE\bfseries}
\subsectionfont{\color{secondaryGray}\Large\bfseries}
\subsubsectionfont{\color{primaryTeal}\large\itshape}

%--- Fancy Boxes for Definitions/Key Points ---
\usepackage{tcolorbox}
\newtcolorbox{conceptbox}[1]{
  colback=boxBack, 
  colframe=primaryTeal, 
  fonttitle=\bfseries, 
  title=#1, 
  arc=1mm
}

%--- Hyperlink Setup ---
\hypersetup{
    colorlinks=true,
    linkcolor=primaryTeal,
    urlcolor=primaryTeal,
    citecolor=secondaryGray
}

%--- Title Info ---
\title{\color{primaryTeal}\textbf{Fluctuations in Inflation}}
\author{\color{secondaryGray}Cosmological Perturbation Theory}
\date{\today}

\begin{document}

\maketitle
\hrule
\vspace{1em}

\begin{abstract}
\noindent \textbf{Note on Units:} Throughout this section, we work in reduced Planck units, where $M_P = 1$.
\end{abstract}

\section{Introduction}
We analyze the decomposition of perturbations of the homogeneous metric into scalar, vector, and tensor categories. 

\textbf{Vector perturbations} are governed by a constraint equation relating the gauge-invariant vector metric perturbation to the divergence-free velocity of the fluid. In the presence of scalar fields, this velocity vanishes, and vector modes decay rapidly. Therefore, we focus exclusively on \textbf{scalar} and \textbf{tensor} perturbations.

\section{Scalar Perturbations}

\subsection{Multi-Field Inflation Action}
We consider the general action $S$ for $n$ scalar fields $\varphi_i$ (where $i=1, \dots, n$):
\begin{equation}
S= \int d^4 x \sqrt{-g}\left[\frac{1}{2} G_{ij}\partial_{\mu}\varphi^i\partial^{\mu}\varphi^j -V(\varphi)\right]
\end{equation}
Here, $V$ is the scalar potential and $G_{ij}$ is the field space metric. We analyze linear perturbations around a homogeneous unperturbed Universe. The field and metric are decomposed as:
\begin{align}
\varphi_i (t,\vec x) &=  \varphi_i (t) + \delta \varphi_i (t,\vec x) \\
g_{\mu \nu }(t,\vec x) &= \bar{g}_{\mu \nu }(t) + h_{\mu \nu }(t,\vec x)
\end{align}

\subsection{Background Equations}
The unperturbed background equations of motion for the $n$ scalar fields are:
\begin{equation}
\ddot{\varphi}^i + \Gamma^{i}_{jk}\dot{\varphi}^j\dot{\varphi}^k + 3H\dot{\varphi}^i + G^{ij}V_{,j} = 0
\label{eq:background_motion}
\end{equation}
The Friedmann equations governing the expansion rate $H$ are:
\begin{equation}
H^2 = \frac{1}{3} \left[ \frac{1}{2} G_{ij}\dot{\varphi}^i\dot{\varphi}^j + V \right], \quad \quad \dot{H} = -\frac{1}{2} G_{ij}\dot{\varphi}^i\dot{\varphi}^j
\end{equation}

\subsection{The Newtonian Gauge}
We adopt the \textbf{Newtonian gauge}, where the scalar metric perturbations $B$ and $E$ vanish. Assuming no anisotropic stress ($\Phi = \Psi$), the perturbed line element is:
\begin{equation}
ds^2 = -(1+2\Psi)dt^2 + a^2(1-2\Psi)\delta_{ij}dx^i dx^j
\end{equation}
where $\Psi$ represents the Bardeen potential.

\subsubsection{Evolution in e-fold Time}
It is often convenient to work with the number of e-folds, $N$, rather than cosmic time $t$, defined by $dN = H dt$. The field equations transform as:
\begin{equation}
\frac{D^2 \varphi^i}{dN^2} + (3-\epsilon)\frac{D \varphi^i}{dN} + \frac{1}{H^2}G^{ij}V_{,j} = 0
\end{equation}
where $\epsilon = -\dot{H}/H^2$.

\subsection{Mukhanov-Sasaki Equation}
To solve for the perturbations, we utilize the gauge-invariant \textbf{Mukhanov-Sasaki variable} $Q^i$, which combines field and metric perturbations:
\begin{equation}
Q^i = \delta \varphi^i + \frac{\dot{\varphi}^i}{H}\Psi
\end{equation}
For a single field, the mode equation for $u_k = -a Q_k$ is:
\begin{equation}
u_k'' + \left( k^2 - \frac{z''}{z} \right)u_k = 0
\end{equation}
where primes denote derivatives with respect to conformal time $\tau$, and $z = a\dot{\varphi}/H$.

\begin{conceptbox}{Initial Conditions: Bunch-Davies Vacuum}
We assume the fields originate in the \textbf{Bunch-Davies vacuum}. In the remote past, modes were deep inside the horizon ($k \gg aH$). The initial condition for the mode function $u_k$ is:
\begin{equation}
u_k(\tau) \to \frac{e^{-ik\tau}}{\sqrt{2k}} \quad \text{as} \quad \tau \to -\infty
\end{equation}
\end{conceptbox}

\section{Power Spectrum}
The primary observable is the power spectrum of the comoving curvature perturbation, $\mathcal{R}$. The dimensionless power spectrum $P_{\mathcal{R}}(k)$ is defined by:
\begin{equation}
\langle \mathcal{R}_{\mathbf{k}} \mathcal{R}_{\mathbf{k}'}^* \rangle = (2\pi)^3 \delta(\mathbf{k}-\mathbf{k}') \frac{2\pi^2}{k^3} P_{\mathcal{R}}(k)
\end{equation}
Analytically,
\begin{equation}
P_{\mathcal{R}}(k) = \frac{k^3}{2\pi^2} |\mathcal{R}_k|^2
\end{equation}
where $\mathcal{R}_k$ is evaluated after the mode exits the horizon ($k \ll aH$). In the slow-roll approximation, this simplifies to:
\begin{equation}
P_{\mathcal{R}} \approx \frac{H^2}{8\pi^2 \epsilon} \bigg|_{k=aH}
\end{equation}

\newpage
\section{Numerical Procedure for Scalar Power Spectrum}
\label{sec:numerical}

While slow-roll approximations are useful, multi-field models often require exact numerical integration. The procedure to derive the exact power spectrum for $n$ scalar fields is as follows:

\begin{enumerate}
    \item \textbf{Setup:} Define the number of scalar fields, $n$, and the field metric $G_{ij}$. If $n > 1$, diagonalize the field metric matrix if necessary.
    
    \item \textbf{Initial Conditions (Background):} Set the initial field values. Calculate initial velocities assuming the attractor solution:
    \begin{equation}
    \dot{\varphi}^i \Big|_{ic} = -\frac{V_{,i}}{3H} \Big|_{ic}
    \end{equation}
    
    \item \textbf{Background Evolution:} Solve the coupled background equations (Eq. \ref{eq:background_motion}) numerically until the end of inflation, defined by the condition $\epsilon_H = 1$.
    
    \item \textbf{Mode Definition:} Select the comoving wavenumber $k$ of interest. It is defined relative to the horizon scale:
    \begin{equation}
    k = C \cdot (aH)_{ic}
    \end{equation}
    where we choose $C \gg 1$ to ensure the mode starts deep inside the horizon (sub-horizon).
    
    \item \textbf{Perturbation Integration:} Solve the system of background equations and perturbation equations simultaneously. 
    \begin{itemize}
        \item Use the \textbf{Bunch-Davies} conditions (Eq. 24 in previous notes) to set initial values for $\delta \varphi$ and $\Psi$.
        \item Integrate until the mode is well outside the horizon ($k \ll aH$) and the solution for the curvature perturbation freezes out.
    \end{itemize}
    
    \item \textbf{Evaluate Observables:} Calculate the curvature perturbation $\mathcal{R}$ and isocurvature modes $\mathcal{S}$ at the end of the integration. Compute the power spectrum using:
    \begin{equation}
    P_{\mathcal{R}} = \frac{k^3}{2\pi^2}|\mathcal{R}_k|^2
    \end{equation}
\end{enumerate}

\end{document}