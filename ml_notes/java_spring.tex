\documentclass[11pt]{article}

% -------------------------------------------------------------
% PACKAGES
% -------------------------------------------------------------
\usepackage[utf8]{inputenc}
\usepackage[T1]{fontenc}
\usepackage{amsmath, amssymb}
\usepackage{graphicx}
\usepackage{geometry}
\usepackage{titlesec}
\usepackage{xcolor}
\usepackage{tcolorbox}
\usepackage{hyperref}
\usepackage{listings}

% -------------------------------------------------------------
% PAGE GEOMETRY
% -------------------------------------------------------------
\geometry{
    a4paper,
    margin=2.5cm
}

% -------------------------------------------------------------
% COLORS
% -------------------------------------------------------------
\definecolor{mainblue}{HTML}{00897B}
\definecolor{lightblue}{HTML}{E6F0FF}
\definecolor{darkgray}{HTML}{2E2E2E}

% -------------------------------------------------------------
% SECTION FORMAT
% -------------------------------------------------------------
\titleformat{\section}{\Large\bfseries\color{mainblue}}{}{0pt}{}
\titleformat{\subsection}{\large\bfseries\color{darkgray}}{}{0pt}{}
\titleformat{\subsubsection}{\normalsize\bfseries\color{darkgray}}{}{0pt}{}

% -------------------------------------------------------------
% CUSTOM BOXES
% -------------------------------------------------------------
\newtcolorbox{infobox}{
    colback=lightblue,
    colframe=mainblue,
    boxrule=0.8pt,
    arc=3pt,
    left=8pt, right=8pt, top=6pt, bottom=6pt
}

\newtcolorbox{ideabox}{
    colback=white,
    colframe=mainblue,
    boxrule=1pt,
    arc=2pt,
    left=8pt, right=8pt, top=6pt, bottom=6pt
}

\newtcolorbox{examplebox}{
    colback=white,
    colframe=darkgray,
    boxrule=0.6pt,
    arc=2pt,
    left=8pt, right=8pt, top=6pt, bottom=6pt
}

% -------------------------------------------------------------
% LISTINGS STYLE (JAVA + SHELL)
% -------------------------------------------------------------
\lstdefinestyle{java}{
    language=Java,
    backgroundcolor=\color{lightblue},
    keywordstyle=\color{mainblue}\bfseries,
    stringstyle=\color{darkgray},
    commentstyle=\color{gray},
    basicstyle=\ttfamily\small,
    frame=single,
    rulecolor=\color{mainblue},
    breaklines=true,
    showstringspaces=false
}

\lstdefinestyle{shell}{
    language=bash,
    backgroundcolor=\color{lightblue},
    keywordstyle=\color{mainblue}\bfseries,
    stringstyle=\color{darkgray},
    commentstyle=\color{gray},
    basicstyle=\ttfamily\small,
    frame=single,
    rulecolor=\color{mainblue},
    breaklines=true,
    showstringspaces=false
}

% -------------------------------------------------------------
% TITLE
% -------------------------------------------------------------
\title{\textbf{\Huge Java Spring Boot Notes}\\[4pt]
\large Commands, Examples, Project Structure \& Base Application}
\author{}
\date{}

% -------------------------------------------------------------
% DOCUMENT START
% -------------------------------------------------------------
\begin{document}

\maketitle
\tableofcontents
\newpage

% =============================================================
\section{What is Spring Boot?}
% =============================================================

\begin{infobox}
\textbf{Spring Boot} is a framework for building Java applications quickly.  

It provides:
\begin{itemize}
    \item opinionated defaults (\emph{``starter'' dependencies}),
    \item embedded server (Tomcat/Jetty),
    \item auto-configuration,
    \item easy packaging as a \texttt{.jar}.
\end{itemize}
\end{infobox}

\begin{ideabox}
Mental model:  
You write Java classes (\texttt{@RestController}, \texttt{@Service}, \texttt{@Entity}),  
Spring Boot wires everything together and runs an HTTP server for you.
\end{ideabox}

% =============================================================
\section{Basic Commands (Maven \& Spring Boot)}
% =============================================================

\subsection{Create a New Project (via Spring Initializr ZIP)}

Often you use the website \texttt{start.spring.io}, download a ZIP, and unzip it.

Alternatively, you can use the CLI (if installed):

\begin{examplebox}
\begin{lstlisting}[style=shell, title={Generate a new project (Spring CLI)}]
spring init \
  --dependencies=web \
  --groupId=com.example \
  --artifactId=demo \
  --name=demo \
  demo-project
\end{lstlisting}
\end{examplebox}

\subsection{Run the Application}

Inside the project root (where \texttt{pom.xml} is):

\begin{examplebox}
\begin{lstlisting}[style=shell, title={Run Spring Boot app with Maven}]
# compile and run with embedded Tomcat on default port 8080
mvn spring-boot:run
\end{lstlisting}
\end{examplebox}

\subsection{Build a \texttt{.jar} File}

\begin{examplebox}
\begin{lstlisting}[style=shell, title={Build executable JAR}]
mvn clean package

# Then run the jar:
java -jar target/demo-0.0.1-SNAPSHOT.jar
\end{lstlisting}
\end{examplebox}

\subsection{Common Maven Commands}

\begin{examplebox}
\begin{lstlisting}[style=shell, title={Useful Maven commands}]
mvn clean           # delete compiled artifacts (target/)
mvn compile         # compile source code
mvn test            # run tests
mvn package         # build JAR/WAR
mvn spring-boot:run # run Spring Boot application
\end{lstlisting}
\end{examplebox}

% =============================================================
\section{Project Structure (Analysis of src/)}
% =============================================================

\subsection{Typical Maven Spring Boot Layout}

\begin{examplebox}
\begin{verbatim}
demo/
  pom.xml
  src/
    main/
      java/
        com/example/demo/
          DemoApplication.java
          controller/
            HelloController.java
          service/
            HelloService.java
          model/
            User.java
      resources/
        application.properties
        static/
        templates/
    test/
      java/
        com/example/demo/
          DemoApplicationTests.java
\end{verbatim}
\end{examplebox}

\subsection{Analysis of Key Files}

\begin{itemize}
    \item \textbf{pom.xml}  
          Maven configuration: dependencies, plugins, Java version, build info.
    \item \textbf{DemoApplication.java}  
          Main class with \texttt{@SpringBootApplication} and \texttt{main()} method.  
          Entry point of the app.
    \item \textbf{controller/HelloController.java}  
          Web layer. Defines HTTP endpoints with \texttt{@RestController}, \texttt{@GetMapping}, etc.
    \item \textbf{service/HelloService.java}  
          Business logic. Annotated with \texttt{@Service}.
    \item \textbf{model/User.java}  
          Data model / entity class (fields, getters, setters, etc.).
    \item \textbf{application.properties}  
          Configuration: server port, DB URL, usernames, etc.
\end{itemize}

\begin{ideabox}
Pattern to remember:  
\textbf{Controller} handles HTTP, \textbf{Service} handles logic, \textbf{Repository} talks to database.
\end{ideabox}

% =============================================================
\section{Making a Base Spring Boot Application}
% =============================================================

\subsection{Main Application Class}

\begin{examplebox}
\begin{lstlisting}[style=java, title={DemoApplication.java}]
package com.example.demo;

import org.springframework.boot.SpringApplication;
import org.springframework.boot.autoconfigure.SpringBootApplication;

// @SpringBootApplication = @Configuration + @EnableAutoConfiguration + @ComponentScan
@SpringBootApplication
public class DemoApplication {

    public static void main(String[] args) {
        // Bootstraps the application, starts Spring context and embedded server
        SpringApplication.run(DemoApplication.class, args);
    }
}
\end{lstlisting}
\end{examplebox}

\subsubsection*{Line-by-line Analysis}

\begin{itemize}
    \item \texttt{package com.example.demo;} – defines the Java package (folder structure).
    \item \texttt{@SpringBootApplication} – tells Spring Boot to:
          \begin{itemize}
              \item search for components in this package and subpackages,
              \item enable auto-configuration,
              \item treat this class as configuration.
          \end{itemize}
    \item \texttt{public static void main(...)} – standard Java entry point.
    \item \texttt{SpringApplication.run(...)} – starts Spring, creates Beans, runs embedded Tomcat.
\end{itemize}

% -------------------------------------------------------------
\subsection{Simple REST Controller}
% -------------------------------------------------------------

\begin{examplebox}
\begin{lstlisting}[style=java, title={HelloController.java}]
package com.example.demo.controller;

import org.springframework.web.bind.annotation.GetMapping;
import org.springframework.web.bind.annotation.RestController;

@RestController // tells Spring this class handles HTTP requests
public class HelloController {

    @GetMapping("/hello")
    public String hello() {
        return "Hello from Spring Boot!";
    }
}
\end{lstlisting}
\end{examplebox}

\subsubsection*{What happens when you run the app?}

\begin{ideabox}
\begin{itemize}
    \item You start the app with \texttt{mvn spring-boot:run}.
    \item Spring Boot starts Tomcat on port \texttt{8080}.
    \item A GET request to \texttt{http://localhost:8080/hello}  
          is routed to \texttt{hello()}.
    \item The method returns a \texttt{String} → sent as HTTP response body.
\end{itemize}
\end{ideabox}

% -------------------------------------------------------------
\subsection{Using a Service Class}
% -------------------------------------------------------------

\begin{examplebox}
\begin{lstlisting}[style=java, title={HelloService.java}]
package com.example.demo.service;

import org.springframework.stereotype.Service;

@Service
public class HelloService {

    public String buildMessage(String name) {
        return "Hello, " + name + "! Welcome to Spring Boot.";
    }
}
\end{lstlisting}
\end{examplebox}

\begin{examplebox}
\begin{lstlisting}[style=java, title={HelloController with Service Injection}]
package com.example.demo.controller;

import com.example.demo.service.HelloService;
import org.springframework.web.bind.annotation.GetMapping;
import org.springframework.web.bind.annotation.RequestParam;
import org.springframework.web.bind.annotation.RestController;

@RestController
public class HelloController {

    private final HelloService helloService;

    // Constructor injection (recommended)
    public HelloController(HelloService helloService) {
        this.helloService = helloService;
    }

    @GetMapping("/hello")
    public String hello(@RequestParam(defaultValue = "World") String name) {
        return helloService.buildMessage(name);
    }
}
\end{lstlisting}
\end{examplebox}

\subsubsection*{Analysis of Dependency Injection}

\begin{itemize}
    \item \texttt{@Service} on \texttt{HelloService} – Spring manages it as a Bean.
    \item Constructor parameter \texttt{HelloService helloService} – Spring injects it automatically.
    \item \texttt{@RequestParam} – reads \texttt{?name=...} from the URL.
\end{itemize}

% =============================================================
\section{Configuration: application.properties}
% =============================================================

\subsection{Basic Settings}

\begin{examplebox}
\begin{verbatim}
# src/main/resources/application.properties

# change port (default 8080)
server.port=8081

# logging level
logging.level.root=INFO

# example datasource (H2 in-memory)
spring.datasource.url=jdbc:h2:mem:testdb
spring.jpa.hibernate.ddl-auto=update
\end{verbatim}
\end{examplebox}

\begin{ideabox}
Spring Boot reads \texttt{application.properties} (or \texttt{application.yml}) on startup  
and automatically configures many components (server, DB, logging, etc.).
\end{ideabox}

% =============================================================
\section{More Java Examples (CRUD Skeleton)}
% =============================================================

\subsection{Simple Model Class}

\begin{examplebox}
\begin{lstlisting}[style=java, title={User.java}]
package com.example.demo.model;

public class User {

    private Long id;
    private String name;
    private String email;

    // Constructors
    public User() {}

    public User(Long id, String name, String email) {
        this.id    = id;
        this.name  = name;
        this.email = email;
    }

    // Getters and setters
    public Long getId() { return id; }
    public void setId(Long id) { this.id = id; }

    public String getName() { return name; }
    public void setName(String name) { this.name = name; }

    public String getEmail() { return email; }
    public void setEmail(String email) { this.email = email; }
}
\end{lstlisting}
\end{examplebox}

\subsection{Controller with Basic CRUD Endpoints (in-memory list)}

\begin{examplebox}
\begin{lstlisting}[style=java, title={UserController.java (in-memory)}]
package com.example.demo.controller;

import com.example.demo.model.User;
import org.springframework.web.bind.annotation.*;

import java.util.ArrayList;
import java.util.List;

@RestController
@RequestMapping("/users")
public class UserController {

    private final List<User> users = new ArrayList<>();

    @GetMapping
    public List<User> getAllUsers() {
        return users;
    }

    @PostMapping
    public User createUser(@RequestBody User user) {
        // simple id assignment for demo
        user.setId((long) (users.size() + 1));
        users.add(user);
        return user;
    }

    @GetMapping("/{id}")
    public User getUser(@PathVariable Long id) {
        return users.stream()
                .filter(u -> id.equals(u.getId()))
                .findFirst()
                .orElse(null);
    }
}
\end{lstlisting}
\end{examplebox}

\subsubsection*{What this demonstrates}

\begin{itemize}
    \item \texttt{@RequestMapping("/users")} – base path for all endpoints in this controller.
    \item \texttt{@GetMapping} – HTTP GET, returns all users or one user.
    \item \texttt{@PostMapping} – HTTP POST, creates a new user.
    \item \texttt{@RequestBody} – maps JSON body → \texttt{User} object.
    \item \texttt{@PathVariable} – maps \texttt{/users/\{id\}} → method parameter.
\end{itemize}

% =============================================================
\section{How to Build Your Own Base Project (Step-by-step)}
% =============================================================

\subsection{Checklist}

\begin{ideabox}
To create a base Spring Boot application:
\begin{enumerate}
    \item Generate project (Spring Initializr or CLI).
    \item Implement \texttt{DemoApplication.java} with \texttt{@SpringBootApplication}.
    \item Add one \textbf{Controller} with \texttt{@RestController}.
    \item Optionally add a \textbf{Service} class and inject it.
    \item Configure basic properties in \texttt{application.properties}.
    \item Run locally with \texttt{mvn spring-boot:run}.
\end{enumerate}
\end{ideabox}

\subsection{Minimal Base Example}

\begin{examplebox}
\begin{lstlisting}[style=java, title={Minimal Base Application: main + controller}]
package com.example.base;

import org.springframework.boot.SpringApplication;
import org.springframework.boot.autoconfigure.SpringBootApplication;
import org.springframework.web.bind.annotation.GetMapping;
import org.springframework.web.bind.annotation.RestController;

@SpringBootApplication
public class BaseApplication {

    public static void main(String[] args) {
        SpringApplication.run(BaseApplication.class, args);
    }

    @RestController
    static class BaseController {

        @GetMapping("/")
        public String index() {
            return "Base Spring Boot app is running!";
        }
    }
}
\end{lstlisting}
\end{examplebox}

\subsubsection*{Why this is nice as a base}

\begin{itemize}
    \item Only one file to start with.
    \item You can later move \texttt{BaseController} to its own package.
    \item You already have a health-like endpoint at \texttt{/}.
\end{itemize}

% =============================================================
\section{Summary}
% =============================================================

\begin{ideabox}
Core ideas to keep in mind:
\begin{itemize}
    \item Spring Boot starts an embedded server and auto-configures a lot for you.
    \item \textbf{Main class} with \texttt{@SpringBootApplication} is the entry point.
    \item \textbf{Controllers} define HTTP endpoints.
    \item \textbf{Services} hold business logic.
    \item \textbf{application.properties} controls behaviour (port, DB, logging, etc.).
    \item Maven commands: \texttt{mvn spring-boot:run}, \texttt{mvn package}, \texttt{java -jar ...}
\end{itemize}
This is enough base to start building real Java Spring Boot applications.
\end{ideabox}

\end{document}
