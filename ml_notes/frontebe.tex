\documentclass[11pt]{article}

% -------------------------------------------------------------
% PACKAGES
% -------------------------------------------------------------
\usepackage[utf8]{inputenc}
\usepackage[T1]{fontenc}
\usepackage{amsmath, amssymb}
\usepackage{graphicx}
\usepackage{geometry}
\usepackage{titlesec}
\usepackage{xcolor}
\usepackage{tcolorbox}
\usepackage{hyperref}
\usepackage{listings}

% -------------------------------------------------------------
% PAGE GEOMETRY
% -------------------------------------------------------------
\geometry{
    a4paper,
    margin=2.5cm
}

% -------------------------------------------------------------
% COLORS
% -------------------------------------------------------------
\definecolor{mainblue}{HTML}{0059B3}
\definecolor{lightblue}{HTML}{E6F0FF}
\definecolor{darkgray}{HTML}{2E2E2E}

% -------------------------------------------------------------
% SECTION FORMAT
% -------------------------------------------------------------
\titleformat{\section}{\Large\bfseries\color{mainblue}}{}{0pt}{}
\titleformat{\subsection}{\large\bfseries\color{darkgray}}{}{0pt}{}
\titleformat{\subsubsection}{\normalsize\bfseries\color{darkgray}}{}{0pt}{}

% -------------------------------------------------------------
% JAVASCRIPT LISTINGS DEFINITION (IMPORTANT FIX)
% -------------------------------------------------------------
\lstdefinelanguage{JavaScript}{
    keywords={typeof, new, true, false, catch, function, return, null, catch,
    switch, var, if, in, while, do, else, case, break, const, let, export, import},
    keywordstyle=\color{mainblue}\bfseries,
    ndkeywords={class, export, boolean, throw, implements, import, this},
    ndkeywordstyle=\color{mainblue}\bfseries,
    identifierstyle=\color{black},
    sensitive=false,
    comment=[l]{//},
    morecomment=[s]{/*}{*/},
    commentstyle=\color{gray},
    stringstyle=\color{darkgray},
    morestring=[b]',
    morestring=[b]"
}

\lstdefinelanguage{bash}{
    keywords={npm, cd, ls, mkdir, rm},
    comment=[l]{\#},
    commentstyle=\color{gray},
    alsoletter={-},
}

% -------------------------------------------------------------
% LISTINGS STYLE
% -------------------------------------------------------------
\lstdefinestyle{js}{
    language=JavaScript,
    backgroundcolor=\color{lightblue},
    basicstyle=\ttfamily\small,
    keywordstyle=\color{mainblue}\bfseries,
    commentstyle=\color{gray},
    stringstyle=\color{darkgray},
    frame=single,
    rulecolor=\color{mainblue},
    breaklines=true,
    showstringspaces=false
}

\lstdefinestyle{shell}{
    language=bash,
    backgroundcolor=\color{lightblue},
    basicstyle=\ttfamily\small,
    keywordstyle=\color{mainblue}\bfseries,
    commentstyle=\color{gray},
    frame=single,
    rulecolor=\color{mainblue},
    breaklines=true,
    showstringspaces=false
}

% -------------------------------------------------------------
% CUSTOM BOXES
% -------------------------------------------------------------
\newtcolorbox{infobox}{
    colback=lightblue,
    colframe=mainblue,
    boxrule=0.8pt,
    arc=3pt,
    left=8pt, right=8pt, top=6pt, bottom=6pt
}
\newtcolorbox{ideabox}{
    colback=white,
    colframe=mainblue,
    boxrule=1pt,
    arc=2pt,
    left=8pt, right=8pt, top=6pt, bottom=6pt
}
\newtcolorbox{examplebox}{
    colback=white,
    colframe=darkgray,
    boxrule=0.7pt,
    arc=3pt,
    left=8pt, right=8pt, top=6pt, bottom=6pt
}

% -------------------------------------------------------------
% TITLE
% -------------------------------------------------------------
\title{\textbf{\Huge Frontend React Notes}\\[4pt]
\large Commands, Examples, Project Structure \& Base Application}
\author{}
\date{}

% -------------------------------------------------------------
\begin{document}

\maketitle
\tableofcontents
\newpage

% =============================================================
\section{What is React?}
% =============================================================

\begin{infobox}
React is a JavaScript library for building user interfaces with reusable components.
\end{infobox}

\begin{ideabox}
React = UI = f(state).  
When state changes → UI updates automatically.
\end{ideabox}

% =============================================================
\section{Commands for React Development}
% =============================================================

\subsection{Create React project using Vite}

\begin{examplebox}
\begin{lstlisting}[style=shell]
npm create vite@latest my-react-app -- --template react
cd my-react-app
npm install
npm run dev
\end{lstlisting}
\end{examplebox}

\subsection{Common npm Commands}

\begin{examplebox}
\begin{lstlisting}[style=shell]
npm install
npm run dev
npm run build
npm run preview
\end{lstlisting}
\end{examplebox}

% =============================================================

% =============================================================
\section{React Project Structure}
% =============================================================

\begin{examplebox}
\begin{verbatim}
src/
  main.jsx              <-- mounts React into <div id="root">
  App.jsx               <-- main application logic

  components/           <-- optional reusable components
    Header.jsx
    TodoList.jsx

  App.css               <-- component-specific styles
  index.css             <-- global baseline styles
\end{verbatim}
\end{examplebox}

\begin{ideabox}
React projects follow a simple pattern:
\begin{itemize}
    \item \texttt{main.jsx} starts the app.
    \item \texttt{App.jsx} contains the user interface.
    \item \texttt{components/} stores reusable UI pieces.
\end{itemize}
\end{ideabox}

% =============================================================
\section{Analysis of main.jsx}
% =============================================================

\begin{examplebox}
\begin{lstlisting}[style=js]
import React from 'react'
import ReactDOM from 'react-dom/client'
import App from './App'

ReactDOM.createRoot(document.getElementById('root')).render(
  <React.StrictMode>
    <App />
  </React.StrictMode>,
)
\end{lstlisting}
\end{examplebox}

\begin{ideabox}
\texttt{main.jsx} loads the React application and injects it into:
\[
\texttt{<div id="root"> ... </div>}
\]
Everything drawn by React appears inside this container.
\end{ideabox}

% =============================================================
\section{Example App.jsx}
% =============================================================

\begin{examplebox}
\begin{lstlisting}[style=js]
import { useState } from 'react'

function App() {
  const [count, setCount] = useState(0)

  return (
    <div>
      <h1>Counter</h1>
      <p>Value: {count}</p>

      <button onClick={() => setCount(count + 1)}>+</button>
      <button onClick={() => setCount(0)}>Reset</button>
    </div>
  )
}

export default App
\end{lstlisting}
\end{examplebox}

\begin{ideabox}
This basic example demonstrates the essential frontend tools:
\begin{itemize}
    \item JSX syntax (HTML-like code inside JavaScript)
    \item \texttt{useState} to store changing values
    \item event handlers such as \texttt{onClick}
\end{itemize}
These will be used in any React frontend.
\end{ideabox}

% =============================================================
\section{Components Example}
% =============================================================

\begin{examplebox}
\begin{lstlisting}[style=js, title=Header.jsx]
function Header({ title }) {
  return (
    <header style={{
      background: '#0059B3',
      color: 'white',
      padding: '1rem'
    }}>
      <h1>{title}</h1>
    </header>
  )
}

export default Header
\end{lstlisting}
\end{examplebox}

\begin{ideabox}
A component accepts \texttt{props} (data passed into it).
This allows reuse of the same UI element in multiple places.
\end{ideabox}

% =============================================================
\section{Todo List Example}
% =============================================================

\begin{examplebox}
\begin{lstlisting}[style=js]
import { useState } from 'react'

function TodoApp() {
  const [todos, setTodos] = useState([])
  const [text, setText] = useState('')

  const addTodo = () => {
    if (!text.trim()) return
    setTodos([...todos, text])
    setText('')
  }

  return (
    <div>
      <input value={text} onChange={e => setText(e.target.value)} />
      <button onClick={addTodo}>Add</button>

      <ul>
        {todos.map((t, i) => (
          <li key={i}>{t}</li>
        ))}
      </ul>
    </div>
  )
}

export default TodoApp
\end{lstlisting}
\end{examplebox}

\begin{ideabox}
This example introduces several important frontend concepts:
\begin{itemize}
    \item arrays stored in state
    \item updating arrays using the spread operator (\texttt{...})
    \item rendering lists with \texttt{.map()}
\end{itemize}
These are fundamental tools in any React-based interface.
\end{ideabox}

% =============================================================
\section{Fetching Data from a Backend}
% =============================================================

\begin{examplebox}
\begin{lstlisting}[style=js]
import { useEffect, useState } from 'react'

function Users() {
  const [users, setUsers] = useState([])

  useEffect(() => {
    fetch('http://localhost:8080/api/users')
      .then(res => res.json())
      .then(data => setUsers(data))
  }, [])

  return (
    <ul>
      {users.map(u => (
        <li key={u.id}>{u.name}</li>
      ))}
    </ul>
  )
}

export default Users
\end{lstlisting}
\end{examplebox}

\begin{ideabox}
\texttt{useEffect} runs code when the component first loads.
It is typically used to:
\begin{itemize}
    \item fetch data from an API
    \item initialize state
    \item perform side effects
\end{itemize}
\end{ideabox}

% =============================================================
\section{Summary}
% =============================================================

\begin{ideabox}
Core React concepts for building a frontend:
\begin{itemize}
    \item Components = independent UI blocks.
    \item JSX = HTML-like syntax merged with JavaScript logic.
    \item \texttt{useState} = remembers values between renders.
    \item \texttt{useEffect} = runs side effects (e.g.\ API calls).
    \item Lists rendered with \texttt{.map()}.
    \item Styling via CSS files.
    \item Project started and served with Vite.
\end{itemize}
These concepts form the basis of all modern React frontends.
\end{ideabox}


\end{document}
